\documentclass[a4paper,twoside=off]{scrbook}
\usepackage{scrlayer-scrpage}

\usepackage{amsmath}
\usepackage{amssymb}
\usepackage[utf8]{inputenc} % Required for including letters with accents
\usepackage{amsthm}

\newcommand{\R}{\mathbb{R}}

\theoremstyle{definition}
\newtheorem{definition}{Definition}[section]

\newtheorem{proposition}{Proposition}[section]

\usepackage{pgfplots}
\pgfplotsset{compat=1.16}

\usepackage{polynom}

% Bibliography
\usepackage[style=numeric,sorting=nyt,sortcites=true,autopunct=true,autolang=hyphen,hyperref=true,abbreviate=false,backref=true,backend=biber]{biblatex}
\addbibresource{bibliography.bib} % BibTeX bibliography file
\defbibheading{bibempty}{}
\usepackage{csquotes}
\usepackage{hyperref}
\title{My Mathematical Journeybook}
\author{Alistair Collins}
\date{July 2020 to ?}

\begin{document}
% \frontmatter
\maketitle
\tableofcontents

%----------------------------------------------------------------------------------------
%	CHAPTER 1
%----------------------------------------------------------------------------------------
\chapter{Introduction}
\section{General}
The idea of this book is to bring together and enhance my mathematical learning journey.  By compiling the steps I take (and having references to the books and papers I read) it should hopefully enable me to learn more (by having to understand it enough to write it down) and synchronise the different areas of study.\\
This process should also enhance my understanding of \LaTeX and how to write a book.
\vfill


%----------------------------------------------------------------------------------------
%	CHAPTER 2
%----------------------------------------------------------------------------------------
\chapter{Calculus}
"A Summary of Calculus" by Dovermann \cite{Dovermann}

\section{Chapter 1: Basic Concepts}

\begin{definition}
Absolute value function
\begin{equation}
    \lvert x \rvert =\begin{cases}
    x, & \text{if $x>0$}\\
    -x, & \text{if $x<0$}\\
    0, & \text{if $x=0$}
  \end{cases}
\end{equation}
\end{definition}

The distance between two points $a$ and $b$ on the real line is $\lvert a-b \rvert$. 
Therefore, ${ \{ x \in \R \, | \, \lvert x-a \rvert < \epsilon \} }$ 
is the set of all real numbers whose distance from $a$ is less than $\epsilon$.  Expressed as an interval this set is $(a - \epsilon , a + \epsilon )$.

\begin{equation}
\begin{gathered}
    \lvert x . y \rvert = \lvert x \rvert . \lvert y \rvert \\
    \lvert x + y \rvert \leq \lvert x \rvert + \lvert y \rvert \\
    \lvert \lvert x \rvert - \lvert y \rvert \rvert \leq \lvert x - y \rvert
\end{gathered}
\end{equation}


\begin{proposition}
Let $f$ be a function and $L$ a real number.  We say that
\begin{equation}
    L = \lim\limits_{x \to a} f(x)
\end{equation}
if for all $\epsilon > 0$ there exists a $\delta > 0$, such that $\lvert f(x) - L \rvert < \epsilon$ whenever $x$ is in the domain of $f$ and $ 0 < \lvert x - a \rvert < \delta$.
\end{proposition}
This reads as $L$ is the limit of $f(x)$ as $x$ approaches $a$.


\begin{proposition}
Assume that the domains of the functions $f(x)$ and $g(x)$ both contain an interval of the form
$(d,a)$ or $(a,e)$ where $d < a < e$.  Suppose that
\begin{equation}
    \lim\limits_{x \to a}f(x) = L \text{   and   } \lim\limits_{x \to a}g(x) = M
\end{equation}
and that $c$ is a constant.  Then
\begin{equation}
    \begin{gathered}
        \lim\limits_{x \to a}(f+g)(x) = M + L \\
        \lim\limits_{x \to a}c(f)(x) = cM \\
        \lim\limits_{x \to a}(f . g)(x) = M . L \\
        \lim\limits_{x \to a}(f/g)(x) = M / L\text{ provided that } L \neq 0. \\
    \end{gathered}
\end{equation}
\end{proposition}

\begin{proposition}\textbf{(Pinching Theorem)}
Assume that the domains of the functions $f(x)$, $g(x)$, and $h(x)$ all contain an interval of the form $(d,a)$ or $(a,e)$ where $d < a < e$ and that $f(x) \leq h(x) \leq g(x)$.  If
\begin{equation}
    \lim\limits_{x \to a}f(x) = L = \lim\limits_{x \to a}g(x)
\end{equation}
then the limit of $h(x)$ exists as x approaches $a$, and it is equal to $L$.
\end{proposition}


\begin{proposition}
    If f(x) is a polynomial, a rational function, or a trigonometric function and f(a) is defined, then
    \begin{equation}
        \lim\limits_{x \to a}f(x) = f(a)
    \end{equation}
\end{proposition}

\medskip

The following limits are important in the calculation of some derivatives:
\begin{equation}
    \begin{gathered}
        \lim\limits_{x \to 0} \frac{\cos x - 1}{x} = 0 \\
        \lim\limits_{x \to 0} \frac{\sin x}{x} = 0 \\
        \lim\limits_{x \to a} \frac{x^{n} - a^{n}}{x - a} = na^{n-1}
    \end{gathered}
\end{equation}
The last assertion can be proved using synthetic division, at least if $n$ is an integer.



%----------------------------------------------------------------------------------------
%	CHAPTER ?
%----------------------------------------------------------------------------------------
\chapter{Other}
\section{Rational Roots Test}
[\url{https://www.purplemath.com/modules/rtnlroot.htm}] \\
Trying to find rational roots of a polynomial.  For instance,
\begin{equation}
    x^{4} + 2x^{3} - 7x^{2} - 8x + 12
\end{equation}
The constant term is 12 and the leading coefficient is 1.  The roots will be found at
(plus-or-minus) (factor of the constant term) / (factor of the leading coefficient), so in this case $\pm 12, 6, 4, 3, 2, 1 = -12, -6, -4, -3, -2, -1, 1, 2, 3, 4, 6, 12$

Note that the following figure illustrates that the roots exist at $-3, -2, 1, 2$:

\begin{tikzpicture}
\begin{axis}[
    axis lines = left,
    axis x line shift=-5,
    xtick distance = 2,
    ytick distance = 1,
    width = 10cm,
    height = 10cm,
    xlabel = $x$,
    ylabel = {$f(x)$},
    ymin = -5,
    ymax = 5,
    xmin = -12,
    xmax = 12,
]
%Below the red parabola is defined
\addplot [
    domain=-12:12, 
    restrict y to domain=-20:20,
    samples=200, 
    color=red,
]
{x^4 + 2*x^3 - 7*x^2 - 8*x + 12};
%\addlegendentry{$x^2 - 2x - 1$}

\end{axis}
\end{tikzpicture}

\section{Synthetic Division}
[\url{https://www.purplemath.com/modules/synthdiv.htm}] \\
Synthetic division is a shorthand, or shortcut, method of polynomial division in the special case of dividing by a linear factor -- and it only works in this case. Synthetic division is generally used, however, not for dividing out factors but for finding zeroes (or roots) of polynomials.

For a particular polynomial, first use the Rational Roots Test to determine likely zeros.  Then use Synthetic Division to check each potential factor for a zero remainder.

For instance, using again $f(x) = x^{4} + 2x^{3} - 7x^{2} - 8x + 12 = 0$
\medskip

\polylongdiv{x^4+2*x^3-7*x^2-8*x+12}{x+3} \\
\medskip
is equivalent to \\
\medskip
\polyhornerscheme[x=-3]{x^4 + 2*x^3 - 7*x^2 - 8*x + 12}\\
\medskip

From this latter equation you can write
\begin{equation}
    x^{4} + 2x^{3} - 7x^{2} - 8x + 12 = (x + 3) \left( x^{3} - x^{2} - 4x + 4 + \frac{0}{x+3} \right)
\end{equation}
where as the numerator of the last term is zero, this equates to a root of the equation.

\pagebreak
\section{Unit Circle}
[\url{http://www.texample.net/tikz/examples/unit-circle/}]

\begin{tikzpicture}[scale=5.3,cap=round,>=latex]
    % draw the coordinates
    \draw[->] (-1.5cm,0cm) -- (1.5cm,0cm) node[right,fill=white] {$x$};
    \draw[->] (0cm,-1.5cm) -- (0cm,1.5cm) node[above,fill=white] {$y$};

    % draw the unit circle
    \draw[thick] (0cm,0cm) circle(1cm);

    \foreach \x in {0,30,...,360} {
            % lines from center to point
            \draw[gray] (0cm,0cm) -- (\x:1cm);
            % dots at each point
            \filldraw[black] (\x:1cm) circle(0.4pt);
            % draw each angle in degrees
            \draw (\x:0.6cm) node[fill=white] {$\x^\circ$};
    }

    % draw each angle in radians
    \foreach \x/\xtext in {
        30/\frac{\pi}{6},
        45/\frac{\pi}{4},
        60/\frac{\pi}{3},
        90/\frac{\pi}{2},
        120/\frac{2\pi}{3},
        135/\frac{3\pi}{4},
        150/\frac{5\pi}{6},
        180/\pi,
        210/\frac{7\pi}{6},
        225/\frac{5\pi}{4},
        240/\frac{4\pi}{3},
        270/\frac{3\pi}{2},
        300/\frac{5\pi}{3},
        315/\frac{7\pi}{4},
        330/\frac{11\pi}{6},
        360/2\pi}
            \draw (\x:0.85cm) node[fill=white] {$\xtext$};

    \foreach \x/\xtext/\y in {
        % the coordinates for the first quadrant
        30/\frac{\sqrt{3}}{2}/\frac{1}{2},
        45/\frac{\sqrt{2}}{2}/\frac{\sqrt{2}}{2},
        60/\frac{1}{2}/\frac{\sqrt{3}}{2},
        % the coordinates for the second quadrant
        150/-\frac{\sqrt{3}}{2}/\frac{1}{2},
        135/-\frac{\sqrt{2}}{2}/\frac{\sqrt{2}}{2},
        120/-\frac{1}{2}/\frac{\sqrt{3}}{2},
        % the coordinates for the third quadrant
        210/-\frac{\sqrt{3}}{2}/-\frac{1}{2},
        225/-\frac{\sqrt{2}}{2}/-\frac{\sqrt{2}}{2},
        240/-\frac{1}{2}/-\frac{\sqrt{3}}{2},
        % the coordinates for the fourth quadrant
        330/\frac{\sqrt{3}}{2}/-\frac{1}{2},
        315/\frac{\sqrt{2}}{2}/-\frac{\sqrt{2}}{2},
        300/\frac{1}{2}/-\frac{\sqrt{3}}{2}}
            \draw (\x:1.25cm) node[fill=white] {$\left(\xtext,\y\right)$};

    % draw the horizontal and vertical coordinates
    % the placement is better this way
    \draw (-1.25cm,0cm) node[above=1pt] {$(-1,0)$}
          (1.25cm,0cm)  node[above=1pt] {$(1,0)$}
          (0cm,-1.25cm) node[fill=white] {$(0,-1)$}
          (0cm,1.25cm)  node[fill=white] {$(0,1)$};
\end{tikzpicture}


%----------------------------------------------------------------------------------------
%	Bibliography 
%----------------------------------------------------------------------------------------
\printbibliography[heading=bibintoc]

%----------------------------------------------------------------------------------------
%	END 
%----------------------------------------------------------------------------------------
\end{document}